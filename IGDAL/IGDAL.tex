\documentclass[a4paper,12pt]{article}

\usepackage[english]{babel}
\usepackage{graphicx}
\usepackage[font={small,it}]{caption}
\usepackage{natbib}
\usepackage{tipa}
\DeclareCaptionFont{tiny}{\tiny}

\author{ Veltzer Doron\\
Department of Linguistics\\
Tel Aviv University\\
Ramat Aviv, Tel Aviv, \underline{Israel} }
\date{\today}

\title{Mechanical Turkish} 

\begin{document}

\section{Mechanical Turkish}

\subsection{Introduction}
The paper proposes Recursive Neural Networks (RNNs) as phonological models. To demonstrate their
effectiveness I revisit \cite{becker_phonological_2009} and \cite{becker_surfeit_2011}, summarize
it's account of Turkish's voicing alternations and criticize it on grounds of implausible
learnability. I then demonstrate how RNNs handle the same phenomenon in a much simpler and
more learnable manner.

\subsection{'surfeit of the stimulus' revisited}
Turkish's accusative high vowel morpheme frequently causes intervocalic voicing. Compare
\textipa{[amatS]} 'purpose' \textipa{[amatSW]} (+acc) to \textipa{[anatS]} 'mother hen'
\textipa{[anadZW]} (+acc). The alternation is lexically specified and unpredictable, However, upon
close inspection of the Turkish lexicon (\cite{inkelas_turkish_2000}) several facts about the
correlation \footnote{All of these also have non trivial mutual correlations (i.e. in order to
predict the probability of alternation in the lexicon it is beneficial to model all the parameters
together)} between these alternations and various environmental features emerge:
\begin{itemize}
  \item Length of the stem greatly increases alternations.
  \item Point Of Articulation (POA) of alternation affects alternations.
  \item High vowels preceding the alternating consonant increase alternations.
  \item Back vowels preceding the alternating consonant increase alternations.
\end{itemize}

When asked to produce the form on nonce words, the speakers showed productive knowledge as to the
mutual pattern of length and POA but little to no (to opposite) knowledge of the equally significant
quality of the preceding vowel.

Becker then offers an OT model which enlists the mechanism of constraint cloning (CC)
(\cite{pater_locus_2007}). when learners detect a conflict between a pair of surface
forms, CC operates by cloning the constraint that caused the conflict. After cloning, constraints subdivide the lexicon among themselves. Repeating into as many sections as required for explaining
speaker behavior. The unlearned preceding vowel quality correlation on the other hand is explained away by simply
claiming that no markedness constraints exist to correlate preceding vowel quality and voicing.

The work ends by asserting that a general statistical learner\footnote{such as the GLA whose results
are presents} will learn all available correlations including the surfeit and in doing so overshoot
the mark by greedily ``outperforming'' native speakers.

\subsection{RNNs as phonological models}

I sketch the relevance of the history of ML to OT grammars from the perceptron model that originally
inspired OT to more contemporary ones recently achievıng across the board state of the art results
in Natural Language Processing (NLP).

First an NN was taught the alternation pattern of the accusative form on the lexicon, after
learning it applies the form to nonce words. The goal of the network design was to generate a model
that could statistically mimic the ``flawed'' human nonce results. NNs, being the full mapping
generalization models they are, had to be manipulated in some way to allow them to exhibit
universal limits inherent in the human speech mechanism, \textbf{I show three general ways to
weaken their powers of inference}. Immersing them in the time dimension by inputing the phonemes
serially (using an RNN), Overloading their task with other tasks and reducing structural connectivity.

Ending I present results of an RNN structure engineered by applying structural limits\footnote{These
are motivated both by the articulators, assuming different lines of command to the vocal
folds/tongue and by autosegmental phonology.} and overloadeding its task to generate all 3 features
required to form articulator commands.

\pagebreak

\bibliographystyle{apalike}
\nocite{*}
\bibliography{IGDAL}

\end{document}